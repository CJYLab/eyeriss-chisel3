\chapter{开发工具及相关技术介绍}

\section{开发工具介绍}
    \subsection{Scala}
        Scala(Scalable Language)是一门多范式的编程语言,集成了面向对象编程、函数式编程和命令式编程的各种特性,同时该语言基于JVM,兼容Java编写的工具包,同时其内置的静态类型极大程度上降低了构建高性能系统的难度,是一门十分灵活、高级的编程语言。
        Scala自2004年被设计出来以后不断发展壮大,截止2019年2月已经发展到2.12.8版本,社区十分活跃,功能十分强大,不断的被大厂应用于产品:
        \begin{itemize}[topsep = 0 pt]
            \setlength{\topsep}{0pt}
            \setlength{\itemsep}{0pt}
            \setlength{\parsep}{0pt}
            \setlength{\parskip}{0pt}
            \setlength{\partopsep}{0pt}
            \item  Twitter宣布他们已经把大部分后端程序从Ruby迁移到Scala
            \item  Wattzon已经公开宣称,其整个平台都已经是基于Scala基础设施编写的
            \item  瑞银集团把Scala用于一般产品中
            \item  Coursera把Scala作为服务器语言使用
            \item  由UCB开发的大数据集群计算平台——Spark使用Scala开发的
        \end{itemize}       
        \begin{lstlisting}[title=Scala Hello world, frame=shadowbox]
            object HelloWorld {
                def main(args: Array[String]) {
                    println("Hello, world!")
                }
            }
        \end{lstlisting}

    \subsection{Chisel3}
        Chisel是一款由UC Berkeley开发并开源的硬件描述语言,支持高度参数化的生成器和分层设计等高级设计方法进行硬件设计。
        需要说明的是Chisel并不是将C转换成RTL(类似HLS),而是货真价实的硬件描述语言。其本身作为工具包内嵌于Scala中。 \\
        Chisel主要特性归纳如下:
        \begin{figure}[h]
            \begin{itemize}[topsep = 0 pt]
                \setlength{\topsep}{0pt}
                \setlength{\itemsep}{0pt}
                \setlength{\parsep}{0pt}
                \setlength{\parskip}{0pt}
                \setlength{\partopsep}{0pt}
                \item 支持抽象的数据类型和接口
                \item 批量连接
                \item 层次化、面向对象、函数式构造
                \item Scala元编程实现了高度参数化
                \item 内置了可配置的标准单元库
                \item 可生成Verilog
                \item 多时钟域设计
                \item 开源,完备的文档,活跃的社区
            \end{itemize}       

            \begin{lstlisting}[title=Chisel Example, frame=shadowbox]
import chisel3._
class MaxN(val n: Int, val w: Int) extends Module {
    private def Max2(x: UInt, y: UInt) = Mux(x > y, x, y)
    val io = IO(new Bundle {
        val ins = Input(Vec(n, UInt(w.W)))
        val out = Output(UInt(w.W))
    })
    io.out := io.ins.reduceLeft(Max2)
}
            \end{lstlisting}
        \end{figure}

    \subsection{Verilator}

\section{异构处理器技术}
    \subsection{ZYNQ架构}
    \subsection{RISC-V}

\section{深度学习技术}
    \subsection{YOLO}
    \subsection{MobieNet}

\section{本章小结}

% \subsection{二级节标题}

% \subsubsection{三级节标题}

% \paragraph{四级节标题}

% \subparagraph{五级节标题}

% \section{脚注}

% Lorem ipsum dolor sit amet, consectetur adipiscing elit, sed do eiusmod tempor
% incididunt ut labore et dolore magna aliqua. Ut enim ad minim veniam, quis
% nostrud exercitation ullamco laboris nisi ut aliquip ex ea commodo consequat.
% Duis aute irure dolor in reprehenderit in voluptate velit esse cillum dolore eu
% fugiat nulla pariatur. Excepteur sint occaecat cupidatat non proident, sunt in
% culpa qui officia deserunt mollit anim id est laborum.
% \footnote{This is a long long long long long long long long long long long long
% long long long long long long long long long long footnote.}
