\chapter{结论}
随着机器学习的发展,在可预见的未来,人工智能技术会越来越方便人类生活的方方面面。目前,人工智能已经使多个领域得到了巨大突破,各行各业也在积极应用人工智能技术来解决
遇到的问题。由于5G的快速发展和普及,在可遇见的未来,物联网行业将迎来飞跃式的发展,小型嵌入式设备将融入人类生活,这将大大增加对低功耗、高性能计算芯片的需求。
通过使用本文提供的PE阵列生成器,可以大大减少CNN计算加速系统开发的时间和复杂度,给嵌入式设备提供一种切实可行的进行AI推断硬件加速的方案。

本文首先对使用的开发工具进行了介绍,其中重点引出了全新的硬件描述语言Chisel3。与HLS、OpenCL等基于高级层次面向软件开发人员的数字电路设计方法不同,Chisel3
更加贴近底层,可以精确设计寄存器的逻辑行为,更加适合进行高性能数字集成电路开发。
其次,秉持着奥卡姆剃刀越简单越可靠的原则,本文构造了多个功能简单的模块,并对涉及行静止计算核心的模块进行了详细的功能、行为介绍,并给出了实际仿真波形。
其中包括:基于FIFO的可变长移位寄存器,用于实现一个容易控制、低资源占用的移位寄存器功能;使用Designware IP优化的SRAM单元,用于存储局部结果;
PE计算单元,用于进行一维卷积计算,是行静止思想中最小执行单元;简易NoC,用于完成大量filter、img数据的快速分发;PE阵列,由简易NoC和多个PE计算单元共同组成,
可以完整计算一次多卷积核、多通道、多图片的卷积操作。

为完成上述工作,最基本的是要有夯实的数字电路开发基础。万变不离其宗,即使使用Chisel3若仍然按照开发软件的思想取开发数字电路是无法成为一名合格的电路工程师的。
其次掌握Scala的基本数据结构和其函数式编程思想。Scala是编程语言中较难入门的语言,因其糅合了面向对象编程、函数式编程等编程思想,不过一旦掌握了其特性,便会发现Scala是目前编程领域,
最为灵活的,是一门非常值得学习的编程语言。正因为如此,UCB选择了使用Scala作为Chisel3的主要开发语言而不是易学易用的Python,GO等。Chisel3作为一门为数不多的硬件描述语言,已经经过了流片验证,
可靠性、稳定性方面无须过多担心,Verilog能完成的,Chisel3只会完成的更加简洁、高校,大大降低编码工作量。

本文的主要思想来源于文献\cite{chen2016eyeriss},作为掀起AI加速浪潮的开始,文献\cite{chen2016eyeriss}中还能很多工作是本文没有涉及的。
本文的主要工作可以归纳为参考文献\cite{chen2016eyeriss}中的数据流模型,学习并利用Chisel3和自己的实现方式,独立完成一个基于行静止思想的通用性卷积计算加速器的设计和验证,旨在缩短CNN中卷积层计算时间,
为物联网领域提供一套完整的CNN推断加速方案。
% \subsection{二级节标题}

% \subsubsection{三级节标题}

% \paragraph{四级节标题}

% \subparagraph{五级节标题}

% \section{脚注}

% Lorem ipsum dolor sit amet, consectetur adipiscing elit, sed do eiusmod tempor
% incididunt ut labore et dolore magna aliqua. Ut enim ad minim veniam, quis
% nostrud exercitation ullamco laboris nisi ut aliquip ex ea commodo consequat.
% Duis aute irure dolor in reprehenderit in voluptate velit esse cillum dolore eu
% fugiat nulla pariatur. Excepteur sint occaecat cupidatat non proident, sunt in
% culpa qui officia deserunt mollit anim id est laborum.
% \footnote{This is a long long long long long long long long long long long long
% long long long long long long long long long long footnote.}
